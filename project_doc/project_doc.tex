\documentclass[a4paper]{article}
\usepackage[utf8]{inputenc}
\usepackage{hyphsubst}
%\HyphSubstIfExists{ngerman-x-latest}{
%  \HyphSubstLet{ngerman}{ngerman-x-latest}}{}
%\HyphSubstIfExists{german-x-latest}{
%  \HyphSubstLet{german}{german-x-latest}}{}
\usepackage{zi4}
\usepackage{amsmath} %for matrices
\usepackage{amssymb} %for element of N / R etc
\usepackage{amsthm} % proof
\usepackage[english]{babel} % Beweis etc
\newcounter{theorems}
\usepackage{natbib} % cites
\usepackage{hyperref}

%\usepackage{mdframed}
\usepackage{enumerate}
\newcommand{\goTo}{\texttt{GoTo}}
\newcommand{\fail}{\texttt{Fail}}
\newcommand{\outputf}{\texttt{Output}}
%\usepackage[hyphenbreaks]{breakurl}

\usepackage{graphicx} %\includegraphics
\usepackage{tikz}
\usetikzlibrary{positioning,fit,patterns}
\usetikzlibrary{shapes}
\usetikzlibrary{arrows}
\usetikzlibrary{arrows.meta}
\usetikzlibrary{positioning,automata} 
\usepackage[all]{xy}
\usepackage{float}
\usepackage{color}
\usepackage{soul} %ul etc
\definecolor{darkgreen}{rgb}{0.0,0.5,0.0}
\definecolor{darkred}{rgb}{0.5,0.0,0.0}
\definecolor{darkyellow}{rgb}{0.5,0.5,0.0}
\definecolor{lightgreen}{rgb}{0.5,1,0.5}
\definecolor{lightgreen2}{rgb}{0.7,0.9,0.7}

\newcommand{\cfbox}[2]{%
    \colorlet{currentcolor}{.}%
    {\color{#1}%
    \fbox{\color{currentcolor}#2}}%
}

\usepackage{listings}
\usepackage{nth}
%\lstset{numbers=left,language=C,frame=lines,commentstyle=\color{darkgreen}\ttfamily,keywordstyle=\color{blue},mathescape,basicstyle=\ttfamily\scriptsize}

\newtheorem{ex}{Example}
\newtheorem*{ex*}{Example}
\newtheorem*{exS}{Example \cite{thwart}}

\newcommand{\matone}{TODO1}
\newcommand{\mattwo}{TODO2}
\title{\textbf{Anti Bicycle Theft}\bigskip\\Documentation}
\author{Kevin Freeman (\matone)\\ Martin Schwarzmaier (\mattwo)\\Georg-August-Universität Göttingen}
\date{\today}
\parindent 0pt
\setlength{\parindent}{0pt}

\begin{document}
\setlength\parindent{0pt}
\maketitle
\begin{center}
	\textbf{Practical Course on Wireless Sensor Networks}
\end{center}\vspace{10em}
\begin{center}
	\begin{tabular}{ll}
	\textbf{Lab Advisor: }&Dr. Omar Alfandi\\
	\textbf{Lab Assistants: }&Arne Bochem, M.Sc.
	\end{tabular}
\end{center}

\begin{figure}[b]
	\centering
	\includegraphics[scale=1]{logo.png} %sehr groß, besser runterskalieren
\end{figure}

\thispagestyle{empty} %Deckblatt, keine Seitenzahl
\newpage

%\renewcommand{\contentsname}{Inhaltsverzeichnis}
\tableofcontents
\thispagestyle{empty} %inhaltsvz, keine Seitenzahl
\newpage

\setcounter{page}{1} %nun Seitennummerierung bei eins beginnen

\section{Introduction}
... bicycle theft, current problem (goettingen e.g.), pro contra to GSM/Bluetooth e.g.

\section{Used Sensorboards and Motes}
We used the standard IRIS motes for everything....
\url{http://www.memsic.com/userfiles/files/Datasheets/WSN/IRIS_Datasheet.pdf}
\subsubsection{Base Station}
Our base station is connected to a computer via USB. On this computer, a SerialForwarder is run in order to establish a possibility to send and receive data to and from the base station. For this project, on the one hand, we have to push \texttt{ID}s of stolen bikes to the base station in order to disseminate the \texttt{ID}s through the network. On the other hand, a collection protocol is implemented to gather information from stolen bikes, like coordinates.\\
\%TODO several picutres here (CLI, serialforwarder)
\subsubsection{Network Node}
The network nodes are completely omittable as they only enlarge the network. So if one wants a great availability of the network more network nodes are needed. These are connected with other network nodes and disseminate and collect the mentioned data to and from the base station.
\subsubsection{Bicycle Mote}
Bicycle motes are connected to each bicycle and equipped with a mts420-cc sensorboard including GPS-antenna.\\
\%TODO picture of stuff\\
Each of them has a unique identifier (\texttt{ID}), which is used on our webview lateron. Here, the user is able to mark his bicycle as stolen. Afterwards, the \texttt{ID} is disseminated through the whole network. If the bicycle comes in range of a network node, it receives a dissemination packets. These are called "pings", as the bicycle knows that the network is in range and available. It will therefore check, if its own \texttt{ID} is marked as stolen. If so, the GPS antenna is powered on and starts collecting data (current runtime, latitude, longitude).\\
After collecting data successfully and receiving another ping, the bicycle mote will dump the data into the network using the collection protocol. The amount of data per packet can be easily changed within the \texttt{./nesC/DataMsg.h}, as seen in Listing \ref{lst:datamsg}.
\begin{lstlisting}[numbers=left, frame=single,language=C, captionpos=b, caption={DataMsg.h, content of packets}, label=lst:datamsg]
//...
#define MAXBIKES 10
#define COORDS_PER_PACKET 2
typedef nx_struct EasyDisseminationMsg 
{
    nx_uint16_t bikes[MAXBIKES];
} EasyDisseminationMsg;
//...
typedef nx_struct EasyCollectionMsg 
{
    nx_uint16_t nodeid;
    nx_uint32_t current_time;
    nx_uint32_t time[COORDS_PER_PACKET];
    nx_uint32_t lat[COORDS_PER_PACKET];
    nx_uint32_t lon[COORDS_PER_PACKET];
} EasyCollectionMsg;
\end{lstlisting}
Each mote has a RAM of 8kB. Therefore, it is possible to store 600 tuples on the bicycle mote RAM. As we are approximately storing one tuple every 3 minutes, it is possible to store the coordinates of the last 30h. This can be extended by saving onto the flash itself, which we left out for future work, as 30h is enough for all our testcases as well as the battery time is limited due to usage of GPS as well. \%TODO batterien haben 3.4k mAh oder so, gps brauch 70 mAh

\section{Network}
\subsection{Protocols}
\subsection{Base Station and Computer Communication}
\subsection{Topology}

\begin{figure}[h!]
\begin{tikzpicture}[node distance=1.5cm]

%CLOUD
\node [cloud, draw,cloud puffs=10,cloud puff arc=120, aspect=2, inner ysep=1em](cl) at (-0.3,4) {Internet};

\node[ellipse, draw, thick, fill=blue!20](bs) at (-1,2){Base Station+PC};

\node[circle, draw, thick, fill=blue!20](n1) at (0,0){Node$_1$} ;
\node[circle, draw, thick, fill=blue!20, right = of n1](n2){Node$_2$};
\node[strike out, draw,rotate=-50,ultra thick, fill=blue!20] at (2,-1.5){obstacle};
\node[circle, draw, thick, fill=blue!20](n3) at (1,-3){Node$_3$};
\node[circle, draw, thick, fill=blue!20](n4) at (4,-3.5){Node$_4$};

\node[](bike) at (-1,-4){\includegraphics[width=.1\textwidth]{bike.png}};

\draw[->, arrows={Triangle-Triangle}] (bs) -- (cl);

\draw[->, arrows={-Triangle}, draw=red, transform canvas={xshift = 0.05cm}] (bs) -- (n1);
\draw[->, arrows={-Triangle}, draw=green, transform canvas={xshift = -0.05cm}] (n1) -- (bs);

\draw[->, arrows={-Triangle}, draw=red, transform canvas={xshift = 0.05cm}] (n1) -- (n3);
\draw[->, arrows={-Triangle}, draw=green, transform canvas={xshift = -0.05cm}] (n3) -- (n1);

\draw[->, arrows={-Triangle}, draw=red, transform canvas={yshift = 0.05cm}] (n1) -- (n2);
\draw[->, arrows={-Triangle}, draw=green, transform canvas={yshift = -0.05cm}] (n2) -- (n1);

\draw[->, arrows={-Triangle}, draw=red, transform canvas={xshift = 0.05cm}] (n2) -- (n4);
\draw[->, arrows={-Triangle}, draw=green, transform canvas={xshift = -0.05cm}] (n4) -- (n2);

\draw[->, arrows={-Triangle}, draw=red, transform canvas={yshift = 0.05cm}] (n3) -- (n4);
\draw[->, arrows={-Triangle}, draw=green, transform canvas={yshift = -0.05cm}] (n4) -- (n3);

\draw[->, arrows={Triangle-}, draw=red, transform canvas={yshift = 0.05cm}] (n3) -- (-0.5,-3.75);
\draw[->, arrows={Triangle-}, draw=green, transform canvas={yshift = -0.05cm}] (-0.5,-3.75) -- (n3);

\filldraw[color=green] (3,2) rectangle (3.5,2.5);
\node at (4.375,2.25)  {Collection};
\filldraw[color=red] (3,1.5) rectangle (3.5,2.0);
\node at (4.7,1.75)  {Dissemination};
\filldraw[color=black] (3,1.0) rectangle (3.5,1.5);
\node at (4.85,1.25)  {SerialForwarder};




\end{tikzpicture}
\end{figure}
- extensible by "nodes"

\subsection{Our Example Topology}

insert picture here

\section{Conclusion}
\section{Future Aspects}

\section{Relevant codes passages... just a dump}
\subsection{base\_station}
Receiving stolen bicycle IDs from PC
\begin{lstlisting}[numbers=left, frame=single,language=C, captionpos=b, caption={DataMsg.h, content of packets}, label=lst:xxx]
//...
for (i=0;i<MAXBIKES;i++)
{
    pkt.bikes[i]=msg->data[i*2]*256+msg->data[i*2+1];
}
//...
\end{lstlisting}

\subsection{base\_station}
Receiving stolen bicycle IDs from PC
\begin{lstlisting}[numbers=left, frame=single,language=C, captionpos=b, caption={DataMsg.h, content of packets}, label=lst:xxx]
#define MAXPOSITIONS 100
//...
uint32_t lons[MAXPOSITIONS];
uint32_t lats[MAXPOSITIONS];
uint32_t times[MAXPOSITIONS];
//reading
atomic
{
    for (i=current_reading_pos;i!=current_writing_pos;i++)
    {
        msg->time[j] = times[i];
        msg->lat[j] = lats[i];
        msg->lon[j] = lons[i];
        times[i]=0;
        lats[i]=0;
        lons[i]=0;
        current_reading_pos++; //we read the value
        if (current_reading_pos==MAXPOSITIONS)
            current_reading_pos=0;

        j++;
        if (j==COORDS_PER_PACKET)
            break;
    }    
}
//writing
atomic
{
    lats[current_writing_pos]=(uint32_t)(lat*1000000);
    lons[current_writing_pos]=(uint32_t)(lon*1000000);
    times[current_writing_pos]=(uint32_t)((call LocalTimeMicro.get())/1000); //3digits ms
    current_writing_pos++;
    if (current_writing_pos==MAXPOSITIONS)
        current_writing_pos=0;
    call Leds.led0Toggle();
}   
//receiving IDs and starting gps if stolen
event void Value.changed() 
{
uint8_t i;
const EasyDisseminationMsg* newVal = call Value.get();
bool found=FALSE;
pkt = *newVal;
for (i=0;i<MAXBIKES;i++)
{
    if (pkt.bikes[i]==secret())
    {  
        stolen=0x01;
        found=TRUE;
        call Leds.led1On();
        if (gps_started==0)
        {
            gps_started=1;
            call Timer.startOneShot(90000); //wait X/1000 secs
            call GpsControl.start();
        }
        else if (gps_started==2) //startDone for gps
        {   
            call Leds.led2Toggle();
            //it is stolen AND received a broadcast 
            //=> DUMP ONE PACKET
            sendMessage(); 
        }
    }
}
if (found==FALSE)
{
    call Leds.led1Off();
    stolen=0x00;
    if (gps_started>1)
    {
        call GpsControl.stop();
        gps_started=0;
    }
}
\end{lstlisting}



\section{References}
Dissemination and collection protocols for TinyOS: \url{http://tinyos.stanford.edu/tinyos-wiki/index.php/Network_Protocols}


%\newpage
%\section{References}\vspace{-2.5em}
%\renewcommand\refname{}
%\nocite{pgraph}
%\nocite{clrs2}
%\nocite{wiki}
%\bibliographystyle{plain}
%\bibliography{src}{}

%
\end{document}
