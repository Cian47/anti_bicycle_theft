% fancytikzposter.tex, version 2.1
% Original template created by Elena Botoeva [botoeva@inf.unibz.it], June 2012
% 
% This file is distributed under the Creative Commons Attribution-NonCommercial 2.0
% Generic (CC BY-NC 2.0) license
% http://creativecommons.org/licenses/by-nc/2.0/ 


\documentclass{a0poster}

\usepackage{fancytikzposter} 

\usepackage[utf8]{inputenc}
\usepackage{hyphsubst}
%\HyphSubstIfExists{ngerman-x-latest}{
%  \HyphSubstLet{ngerman}{ngerman-x-latest}}{}
%\HyphSubstIfExists{german-x-latest}{
%  \HyphSubstLet{german}{german-x-latest}}{}
\usepackage{zi4}
\usepackage{amsmath} %for matrices
\usepackage{amssymb} %for element of N / R etc
\usepackage{amsthm} % proof
\usepackage[english]{babel} % Beweis etc
\newcounter{theorems}
\usepackage{natbib} % cites
\usepackage{hyperref}

%\usepackage{mdframed}
\usepackage{enumerate}
%\usepackage[hyphenbreaks]{breakurl}

\usepackage{graphicx} %\includegraphics
\usepackage{tikz}
\usetikzlibrary{positioning,fit,patterns}
\usetikzlibrary{shapes}
\usetikzlibrary{arrows}
\usetikzlibrary{arrows.meta}
\usetikzlibrary{positioning,automata} 
\usepackage[all]{xy}
\usepackage{float}
\usepackage{color}
\usepackage{soul} %ul etc
\definecolor{darkgreen}{rgb}{0.0,0.5,0.0}
\definecolor{darkred}{rgb}{0.5,0.0,0.0}
\definecolor{darkyellow}{rgb}{0.5,0.5,0.0}
\definecolor{lightgreen}{rgb}{0.5,1,0.5}
\definecolor{lightgreen2}{rgb}{0.7,0.9,0.7}

\newcommand{\cfbox}[2]{%
    \colorlet{currentcolor}{.}%
    {\color{#1}%
    \fbox{\color{currentcolor}#2}}%
}

\usepackage{listings}
\usepackage{nth}
%\lstset{numbers=left,language=C,frame=lines,commentstyle=\color{darkgreen}\ttfamily,keywordstyle=\color{blue},mathescape,basicstyle=\ttfamily\scriptsize}

\newtheorem{ex}{Example}
\newtheorem*{ex*}{Example}
\newtheorem*{exS}{Example \cite{thwart}}

\newcommand{\matone}{TODO1}
\newcommand{\mattwo}{TODO2}
\newcommand{\mts}{MTS420cc }
\title{\textbf{Anti Bicycle Theft}\bigskip\\Documentation}
\author{Kevin Freeman (\matone)\\ Martin Schwarzmaier (\mattwo)\\Georg-August-Universität Göttingen}
\date{\today}

%%%%% --------- Change here if you want ---------- %%%%%
%% margin for the geometry package, must be changed before using the geometry package
%% default value is 4cm
% \setmargin{4}

%% the space between the blocks
%% default value is 2cm
% \setblockspacing{2}

%% the height of the title stripe in block nodes, decrease it to save space
%% default value is 3cm
% \setblocktitleheight{3}

%% the number of columns in the poster, possible values 2,3
%% default value is 2
% \setcolumnnumber{3}

%% the space between two or more groups of authors from different institutions
%% used in \maketitle
% \setinstituteshift{10}

%% which template to use
%% N1 simple, standard look, with a colored background and gray boxes
%% N2 board with nodes
%% N3 another standard look
%% N4 envelope-like look
%% N5 with a wave-like head, original idea taken from
%%%% http://fc09.deviantart.net/fs71/f/2010/322/1/1/scientific_poster_by_nabuy-d333ria.jpg
%\usetemplate{6}

%% components of the templates
%% (the maximal possible numbers are mentioned as the parameters)
% \usecolortemplate{4}
% \usebackgroundtemplate{5}
% \usetitletemplate{2}
% \useblocknodetemplate{5}
% \useplainblocktemplate{4}
% \useinnerblocktemplate{2}


%% the height of the head drawing on top 
%% applicable to templates N3, 4 and 5
% \setheaddrawingheight{14}


%% change the basic colors
%\definecolor{myblue}{HTML}{008888} 
%\setfirstcolor{myblue}% default 116699
%\setsecondcolor{gray!80!}% default CCCCCC
%\setthirdcolor{red!80!black}% default 991111

%% change the more specific colors
% \setbackgrounddarkcolor{colorone!70!black}
% \setbackgroundlightcolor{colorone!70!}
% \settitletextcolor{textcolor}
% \settitlefillcolor{white}
% \settitledrawcolor{colortwo}
% \setblocktextcolor{textcolor}
% \setblockfillcolor{white}
% \setblocktitletextcolor{colorone}
% \setblocktitlefillcolor{colortwo} %the color of the border
% \setplainblocktextcolor{textcolor}
% \setplainblockfillcolor{colorthree!40!}
% \setplainblocktitletextcolor{textcolor}
% \setplainblocktitlefillcolor{colorthree!60!}
% \setinnerblocktextcolor{textcolor}
% \setinnerblockfillcolor{white}
% \setinnerblocktitletextcolor{white}
% \setinnerblocktitlefillcolor{colorthree}




%%% size of the document and the margins
%% A0
% \usepackage[margin=\margin cm, paperwidth=118.9cm, paperheight=84.1cm]{geometry} 
\usepackage[margin=\margin cm, paperwidth=84.1cm, paperheight=118.9cm]{geometry}
%% B1
% \usepackage[margin=\margin cm, paperwidth=70cm, paperheight=100cm]{geometry}



%% changing the fonts
\usepackage{cmbright}
%\usepackage[default]{cantarell}
%\usepackage{avant}
%\usepackage[math]{iwona}
\usepackage[math]{kurier}
\usepackage[T1]{fontenc}


%% add your packages here
\usepackage{hyperref}





\title{Anti Bicycle Theft}
\author{Kevin Freeman, Martin Schwarzmaier\\
  Practical Course on Wireless Sensor Networks\\
  \texttt{Advised by: Dr. Omar Alfandi; Arne Bochem, M.Sc.}
}


\begin{document}

%%%%% ---------- the background picture ---------- %%%%%
%% to change it modify the macro \BackgroundPicture
\ClearShipoutPicture
\AddToShipoutPicture{\BackgroundPicture}

\noindent % to have the picture right in the center
\begin{tikzpicture}
  \initializesizeandshifts
  % \setxshift{15}
  % \setyshift{2}


  %% the title block, #1 - shift, the default value is (0,0), #2 - width, #3 - scale
  %% the alias of the title block is `title', so we can refer to its boundaries later
  \ifthenelse{\equal{\template}{1}}{ 
    \titleblock{47}{1}
  }{
    \titleblock{47}{1.5}
  }

  %% a logo can be added to the title block
  %% #1 - anchor relative to the title block, #2 - shift, #3 - width, #3 - file name
  % \ifthenelse{\equal{\template}{2}}{ 
  %   \addlogo[south west]{(2,0)}{6cm}{unibz_b.png}
  % }{
  %   \addlogo[south west]{(2,0)}{6cm}{unibz_w.png}
  % }


  %% a block node, with the specified position (optional), title and the content
  %% #1 - where (optional), #2 - title, #3 - text
  %%%%%%%%%% ------------------------------------------ %%%%%%%%%%
  \blocknode%
  {Introduction}%
  {
Bicycle theft is a major problem in many cities, especially university towns.
Our aim was to create a system that enables a bike owner to track a stolen bike. While other systems already on the market are trying to accomplish this by using GSM modules we tried to use a minimalistic approach:
A tracking GPS module that can be turned on when a bike has been stolen, tracks positions and retransmits them to the user via a short distance wireless interface.
  
  Start with the following document:

    \coloredbox{colorthree!50!}{
      ...
    }
  }


  %% a callout block
  %% #1 - rotate angle (optional), #2 - from, #3 - where, #4 - width, #5 - text
  %%%%%%%%%% ------------------------------------------ %%%%%%%%%%
  \calloutblock{($(box.center)+(-2,-8)$)}
  {($(box.center)+(10,-1)$)}
  {19cm}
  {\small
    Macro for creating a block node:
    \begin{itemize}
    \item[] \textbackslash blocknode\{Block Title\}\{Block Content\}
    \end{itemize}
    Macro \textbackslash blocknode has three parameters. The first one is
    optional and it is the position of the block. The first block will be
    automatically placed to (\$(firstrow)-(xshift)-(yshift)\$), which is the
    left corner below the title block. In most of the templates, (firstrow) is
    set to (title.south), where \emph{title} is the alias for the title
    block. Each subsequent block is automatically placed to
    [(\$(box.south)-(yshift)\$)], i.e., below the previous block aliased
    \emph{box}.  You can also use an explicit parameter, e.g., $(-10,30)$ (note
    that (0,0) is the center of the poster). The second parameter is the title
    of the block. Finally, the last parameter is the  actual content. 
  }




  %% by default, the position of the new block node is right below the previous
  %% block node, stored in (currenty)
  %% box is the alias of the previous block, so we can refer to its boundaries

  %%%%%%%%%% ------------------------------------------ %%%%%%%%%%
  \blocknode{Used hardware / environment / Software}%
  {
The system we proposed an implemented prototypically allows a user to remotely turn on a location tracking device, collect data over extended periods of time. This works even if his bike is not in range of a network after it has been activated. Using a easy to use webinterface bikes and be quickly reported as stolen and location data can be conveniently displayed on a digital map.\\
Future work is needed to determine how battery usage can be minimized (e.g. by replacing GPS with WiFi positioning) and whether a sufficient relay network can be established at reasonable costs.  
  
  
  }
  
  %%%%%%%%%% ------------------------------------------ %%%%%%%%%%
  \blocknode{System Description}%
  {

The system consits of a PC running a webserver, a iris mote connected to the PC via USB, a set of relay IRIS motes and so called bike iris motes. The process begins with a user marking his bike as stolen using the system‘s webapp using an ID. The webserver will relay this ID to the basestation which broadcasts it to all relay motes via dissemination. When a bike passes a relay mote it will establish a connection and check whether its ID was disseminated. If this was the case the bike mote will turn on its GPS module and start recording its position in combination with a timestamp.\\
The next time the bike passes a relay mote the position data will be transfered, collected at the base station and made available to the user on the webapp
  
  
  }
  
  %%%%%%%%%% ------------------------------------------ %%%%%%%%%%
  \blocknode{Topology}%
  {

\begin{center}
\begin{tikzpicture}[node distance=1.5cm]

%CLOUD
\node [cloud, draw,cloud puffs=10,cloud puff arc=120, aspect=2, inner ysep=1em](cl) at (-0.3,4) {Internet};

\node[ellipse, draw, thick, fill=blue!20](bs) at (-1,2){Base Station};

\node[circle, draw, thick, fill=blue!20](n1) at (0,0){Node$_1$} ;
\node[circle, draw, thick, fill=blue!20, right = of n1](n2){Node$_2$};
\node[strike out, draw,rotate=-50,ultra thick, fill=blue!20] at (2,-1.5){obstacle};
\node[circle, draw, thick, fill=blue!20](n3) at (1,-3){Node$_3$};
\node[circle, draw, thick, fill=blue!20](n4) at (4,-3.5){Node$_4$};

\node[rotate=0](bike) at (-1,-4){\includegraphics[width=.1\textwidth]{bike.png}};

\draw[->, arrows={Triangle-Triangle}] (bs) -- (cl);
\node at (-0.4,2.75){PC} ;

\draw[->, arrows={-Triangle}, draw=red, transform canvas={xshift = 0.05cm}] (bs) -- (n1);
\draw[->, arrows={-Triangle}, draw=green, transform canvas={xshift = -0.05cm}] (n1) -- (bs);

\draw[->, arrows={-Triangle}, draw=red, transform canvas={xshift = 0.05cm}] (n1) -- (n3);
\draw[->, arrows={-Triangle}, draw=green, transform canvas={xshift = -0.05cm}] (n3) -- (n1);

\draw[->, arrows={-Triangle}, draw=red, transform canvas={yshift = 0.05cm}] (n1) -- (n2);
\draw[->, arrows={-Triangle}, draw=green, transform canvas={yshift = -0.05cm}] (n2) -- (n1);

\draw[->, arrows={-Triangle}, draw=red, transform canvas={xshift = 0.05cm}] (n2) -- (n4);
\draw[->, arrows={-Triangle}, draw=green, transform canvas={xshift = -0.05cm}] (n4) -- (n2);

\draw[->, arrows={-Triangle}, draw=red, transform canvas={yshift = 0.05cm}] (n3) -- (n4);
\draw[->, arrows={-Triangle}, draw=green, transform canvas={yshift = -0.05cm}] (n4) -- (n3);

\draw[->, arrows={Triangle-}, draw=red, transform canvas={yshift = 0.05cm}] (n3) -- (-0.5,-3.75);
\draw[->, arrows={Triangle-}, draw=green, transform canvas={yshift = -0.05cm}] (-0.5,-3.75) -- (n3);

\filldraw[color=green] (3-1,2+0.5) rectangle (3.5-1,2.5+0.5);
\node at (4.375-1,2.25+0.5)  {Collection};
\filldraw[color=red] (3-1,1.5+0.5) rectangle (3.5-1,2.0+0.5);
\node at (4.7-1,1.75+0.5)  {Dissemination};
\filldraw[color=black] (3-1,1.0+0.5) rectangle (3.5-1,1.5+0.5);
\node at (4.85-1,1.25+0.5)  {SerialForwarder};




\end{tikzpicture}
\end{center}
%\caption{Example network including protocols used}\label{fig:topo}

  
  }  
  %%%%%%%%%% ------------------------------------------ %%%%%%%%%%
  \blocknode{Conclusion}%
  {
The system we proposed an implemented prototypically allows a user to remotely turn on a location tracking device, collect data over extended periods of time. This works even if his bike is not in range of a network after it has been activated. Using a easy to use webinterface bikes and be quickly reported as stolen and location data can be conveniently displayed on a digital map.\\
Future work is needed to determine how battery usage can be minimized (e.g. by replacing GPS with WiFi positioning) and whether a sufficient relay network can be established at reasonable costs.  
  
  
  }
  

  %%%%%%%%%% ------------------------------------------ %%%%%%%%%%
  \blocknodew[($(currenty)-(3.5,0)$)]{30}{Variable Width Block Nodes} %
  { You can also create blocks of arbitrary width
    \begin{itemize}
    \item[] \textbackslash blocknodew[coordinate]\{Block width\}\{Block Title\}%
      \{Block Content\}
    \end{itemize} 
    % 
    In this case it is better to specify coordinate manually if you want to have
    blocks aligned vertically. \\

    Note that (xshift) and (yshift) are coordinates created in macro
    \textbackslash initializesizeandshifts, and they allow to have relative
    positioning of block nodes in an automatic fashion. If you want to define
    your own shifts, set new values for (xshift) and (yshift) using commands
    \textbackslash setxshift and \textbackslash setyshift.\\

    Also, it might be useful to know the y-coordinate of the south border of the
    previous block. You can retrieve it by using the command
    \begin{itemize}
    \item[] \textbackslash getcurrentrow\{box\} or \textbackslash getcurrentrow\{note\}
    \end{itemize}
    This coordinate will be stored in (currentrow), which can be used to
    specify the location of the next block node.
  }


  %%%%%%%%%% ------------------------------------------ %%%%%%%%%%
  \plainblock[5]{($(currenty)+(4,2)$)}{35}{fancyTikZposter template} %
  {
    
    \vspace{0.3cm}
    It is a template for scientific posters based on a0poster and TikZ
    only. The current version contains five (plus one) different templates (see my
    posters
    % 
    \href{http://www.inf.unibz.it/~ebotoeva/presentations/abcrs-KR-12-poster.pdf}{%
      \underline{here}} and
    % 
    \href{http://www.inf.unibz.it/~ebotoeva/presentations/boto-RR-12-poster.pdf}{%
      \underline{here}}). The sources of this pdf file can be found
    \href{http://www.inf.unibz.it/~ebotoeva/tikz/tikzposter_sources.zip}{\underline{here}}.}

  


  %%%%%%%%%%%%% NEW COLUMN %%%%%%%%%%%%%%% 
  \startsecondcolumn 

  %%%%%%%%%% ------------------------------------------ %%%%%%%%%%
  \blocknode%
  {Block Nodes in the Second Column}%
  {To start the second column or the third column use commands
    \begin{itemize}
    \item[] \textbackslash startsecondcolumn, and \textbackslash startthirdcolumn.
    \end{itemize}
    If the number of columns is 2, then the last command will not have
    effect. \\

    You can also start a new column with an arbitrary x-coordinate by specifying
    explicitly the coordinate of the new block node as follows:
    \begin{itemize}
    \item[] \textbackslash blocknode[(\$(firstrow)-(yshift)+(x,0)\$)]\{Block
      Title\}\{Block Content\}
    \end{itemize}

    % 
  }


  
  %%%%%%%%%% ------------------------------------------ %%%%%%%%%%
  \blocknode{Useful Macro Within Block Nodes}%
  {There are three types of colored boxes/blocks that you can use inside block
    nodes to highlight information. \\
    
    \begin{tabular}[t]{ll}
      \begin{minipage}{0.5\linewidth}
        \innerblock{Theorem} {Statement}
      \end{minipage}
      & 
      \textbackslash innerblock\{Theorem\}\{Statement\}\\

      \begin{minipage}{0.5\linewidth}
        \innerblockplain[colorone!80!]{Text}
      \end{minipage}
      &
      \textbackslash innerblockplain[colorone!80!]\{Text\}\\ 

      \begin{minipage}{0.5\linewidth}
        \coloredbox{colorthree!50!}{Text}
      \end{minipage}
      &
      \textbackslash coloredbox\{colorthree!50!\}\{Text\}
    \end{tabular}

    \vspace{0.5cm}
    The default figure environment does not work within a tikzpicture. I created
    a new figure environment that can be used instead, based on the code sent by
    Stephan Thober.
    \begin{itemize}
    \item[] \textbackslash begin\{tikzfigure\}[Caption]\\
      \ldots\\
      \textbackslash end\{tikzfigure\}
    \end{itemize} 
    % 

    \begin{tikzfigure}[A shaded circle]
      \begin{tikzpicture}
        \draw[draw=none,inner color=colorthree, outer color=colorone] (0,0) circle (2cm);
      \end{tikzpicture}
    \end{tikzfigure}
  }


  %%%%%%%%%% ------------------------------------------ %%%%%%%%%%
  \calloutblock{($(box.south east)-(8,-2)$)}
  {($(box.south east)-(16,2)$)}
  {30cm}
  {
    There are also callout blocks that allow for a more interesting layout of the
    poster. 
    \begin{itemize}
    \item[] \textbackslash calloutblock[rotate angle]\{from
      coordinate\}\{coordinate\}\{Block Width\}\{Block Content\} 
    \end{itemize}
    The alias for such blocks is \emph{note}.
  }


  %% to place the next node centered vertically in the second column, we can
  %% obtain the y-coordinate of the previous node using macro
  %% \getcurrentrow{note}, where note is the alias of the callout node, and
  %% then specify the coordinate of the next node using coordinate (currentrow)
  \getcurrentrow{note}


  %% a plain block
  %% #1 - rotate angle (optional), #2 - where, #3 - width, #4 - title, #5 - text
  %%%%%%%%%% ------------------------------------------ %%%%%%%%%%
  \plainblock{($(currentrow)+(xshift)-(yshift)$)}%[($(currenty)+(0,10)$)]%
  {32}{Plain blocks} %
  {These blocks are similar to callout blocks. They allow for specifying the
    title of the block.
    \begin{itemize}
    \item[] \textbackslash plainblock[rotate angle]\{coordinate\}\{Block Width\}\{Block
      Title\}\{Block Content\} 
    \end{itemize}
  }


 
  %% the coordinate (currenty) is used in the default placing of the next blocknode
 \getcurrentrow{note}
 \coordinate (currenty) at ($(currentrow)+(xshift)-(yshift)$);



   %%%%%%%%%%%%% NEW COLUMN %%%%%%%%%%%%%%% 
  %% (if column number is 3)
  \startthirdcolumn

  %%%%%%%%%% ------------------------------------------ %%%%%%%%%%
  \blocknode {Personalizing the Poster}%
  {It is possible to adjust the layout of the poster. To impose your own
    setting, you can use these macros:
    \begin{itemize}
    \item Macros for changing sizes
      \begin{itemize}
      \item[] \textbackslash setmargin\{4\},
        %% the height of the head drawing on top
        %% applicable to templates N2 and 4
        \textbackslash setheaddrawingheight\{14\},
        %% the space between two or more groups of authors from different
        %% institutions
        %% used in \maketitle
        \textbackslash setinstituteshift\{10\},\\
        %% the space between the blocks
        %% default value is 2cm
        \textbackslash setblockspacing\{2\},
        %% the height of the title stripe in block nodes, decrease it to save space
        %% default value is 3cm
        \textbackslash setblocktitleheight\{3\}
      \end{itemize}

    \item Other structural macros
      \begin{itemize}
      \item[]  %% the number of columns in the poster, possible values 2,3
        %% default value is 2
        \textbackslash setcolumnnumber\{3\},
        %% which template to use 
        %% N1 simple, standard look, with a colored background and gray boxes
        %% N2 board with nodes
        %% N3 another standard look
        %% N4 envelope like look
        %% N5 with a wave-like head, original idea taken from
        %%%% http://fc09.deviantart.net/fs71/f/2010/322/1/1/scientific_poster_by_nabuy-d333ria.jpg
        %% N6 experimental, oriental style, largely based on template N3
        \textbackslash usetemplate\{6\},\\
        \textbackslash usecolortemplate\{4\},
        \textbackslash usebackgroundtemplate\{5\},
        \textbackslash usetitletemplate\{2\},\\
        \textbackslash useblocknodetemplate\{5\},
        \textbackslash useinnerblocktemplate\{3\},
        \textbackslash useplainblocktemplate\{4\}

      \end{itemize}

    \item Macro for adding logos to the title block
      \begin{itemize}
      \item[] \textbackslash addlogo[south west]\{(0,0)\}\{6cm\}\{filename\}
      \end{itemize}

    \item Macros for the basic colors
      \begin{itemize}
      \item[] \textbackslash setfirstcolor\{green!70!\}, % default 116699
        \textbackslash setsecondcolor\{gray!80!\}, % default CCCCCC
        \textbackslash setthirdcolor\{red!80!black\}% default 991111
      \end{itemize}

    \item Macros for specific colors:
      \begin{itemize}
      \item[] \textbackslash setbackgrounddarkcolor\{colorone!70!black\},
        \textbackslash setbackgroundlightcolor\{{\small colorone!70!}\},\\
        \textbackslash settitletextcolor\{textcolor\},
        \textbackslash settitlefillcolor\{white\},
        \textbackslash settitledrawcolor\{colortwo\},\\
        \textbackslash setblocktextcolor\{textcolor\},
        \textbackslash setblockfillcolor\{white\},\\
        \textbackslash setblocktitletextcolor\{colorone\},
        \textbackslash setblocktitlefillcolor\{colortwo\}, \\
        \textbackslash setplainblocktextcolor\{textcolor\},
        \textbackslash setplainblockfillcolor\{colorthree!40\},\\
        \textbackslash setplainblocktitletextcolor\{textcolor\},
        \textbackslash setplainblocktitlefillcolor\{colorthree!60\}, \\
        \textbackslash setinnerblocktextcolor\{textcolor\},
        \textbackslash setinnerblockfillcolor\{white\},\\
        \textbackslash setinnerblocktitletextcolor\{white\},
        \textbackslash setinnerblocktitlefillcolor\{colorthree\},
      \end{itemize}
    \end{itemize}
  }



\end{tikzpicture}


\end{document}




